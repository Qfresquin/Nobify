\documentclass[a4paper,12pt]{report}
\usepackage[utf8]{inputenc}
\usepackage[T1]{fontenc}
\usepackage[brazil]{babel}
\usepackage{geometry}
\usepackage{titlesec}
\usepackage{listings}
\usepackage{xcolor}
\usepackage{hyperref}
\usepackage{enumitem}
\usepackage{fancyhdr}

% Configuração de margens
\geometry{left=2.5cm, right=2.5cm, top=2.5cm, bottom=2.5cm}

% Configuração de cores para código
\definecolor{codegreen}{rgb}{0,0.6,0}
\definecolor{codegray}{rgb}{0.5,0.5,0.5}
\definecolor{codepurple}{rgb}{0.58,0,0.82}
\definecolor{backcolour}{rgb}{0.95,0.95,0.92}
\definecolor{codeblue}{rgb}{0.1,0.1,0.8}

\lstdefinestyle{mystyle}{
    backgroundcolor=\color{backcolour},   
    commentstyle=\color{codegreen},
    keywordstyle=\color{codeblue}\bfseries,
    numberstyle=\tiny\color{codegray},
    stringstyle=\color{codepurple},
    basicstyle=\ttfamily\footnotesize,
    breakatwhitespace=false,         
    breaklines=true,                 
    captionpos=b,                    
    keepspaces=true,                 
    numbers=left,                    
    numbersep=5pt,                  
    showspaces=false,                
    showstringspaces=false,
    showtabs=false,                  
    tabsize=2,
    frame=single,
    language=C
}

\lstset{style=mystyle}

% Cabeçalho e Rodapé
\pagestyle{fancy}
\fancyhf{}
\rhead{\today}
\lhead{Análise Arquitetural: Transpiler Evaluator}
\rfoot{Página \thepage}

\title{\textbf{Relatório de Auditoria Arquitetural}\\ \large Análise de Violações de Design no Módulo \texttt{transpiler\_evaluator.inc.c}}
\author{Equipe de Desenvolvimento}
\date{\today}

\begin{document}

\maketitle

\tableofcontents
\newpage

\chapter{Introdução}
Este relatório apresenta uma análise crítica do arquivo \texttt{transpiler\_evaluator.inc.c} dentro do projeto de transpilação CMake para NOB. A análise foca em princípios de Engenharia de Software, especificamente o \textbf{Princípio da Responsabilidade Única (SRP)} e \textbf{Separação de Preocupações (SoC)}.

Atualmente, o avaliador atua como um \textit{God Object} (Objeto Deus), acumulando responsabilidades que variam desde a interpretação da AST até manipulação de baixo nível de arquivos, gerenciamento de memória do modelo de dados e execução de processos do sistema operacional.

\chapter{Acoplamento com o Modelo de Dados}
\section{Violação de Encapsulamento (Feature Envy)}
O avaliador acessa e modifica diretamente as estruturas internas de \texttt{Build\_Model} e \texttt{Build\_Target}. Isso cria um acoplamento rígido, onde qualquer alteração na estrutura de dados quebra a lógica de avaliação.

\subsection{Problemas Identificados}
\begin{itemize}
    \item \textbf{Acesso Direto a Campos:} O avaliador define flags como \texttt{target->win32\_executable = true} diretamente.
    \item \textbf{Gestão de Memória Externa:} O avaliador invoca \texttt{arena\_da\_reserve} para arrays que pertencem ao modelo (ex: \texttt{Custom\_Command}, \texttt{CPack}).
    \item \textbf{Lógica de Negócio do Modelo:} A decisão de onde armazenar uma propriedade (se em \texttt{properties[CONFIG\_DEBUG]} ou global) é feita no avaliador, não no modelo.
\end{itemize}

\subsection{Exemplo de Código Problemático}
\begin{lstlisting}[caption={Evaluator manipulando estruturas internas do Model}]
// transpiler_evaluator.inc.c
if (mode_target) {
    // O evaluator sabe alocar memoria dentro do target
    if (!arena_da_reserve(model->arena, 
        (void**)&target->pre_build_commands, ...)) return NULL;
    
    // O evaluator preenche campos internos manualmente
    Custom_Command *cmd = &target->pre_build_commands[*count];
    cmd->command = command;
    cmd->echo = true;
}
\end{lstlisting}

\chapter{Abstração de Sistema e I/O}
O avaliador mistura a lógica de interpretação da linguagem CMake com chamadas de sistema de baixo nível.

\section{Manipulação de Arquivos}
Comandos como \texttt{file(COPY...)} e \texttt{configure\_file(...)} contêm implementações completas de algoritmos de cópia recursiva e leitura de diretórios.

\begin{itemize}
    \item \textbf{Recursive Delete/Copy:} Funções como \texttt{file\_delete\_path\_recursive} e \texttt{file\_copy\_entry\_to\_destination} são utilitários de sistema de arquivos agnósticos à linguagem CMake e não deveriam residir no avaliador.
    \item \textbf{Criação de Diretórios:} A lógica de \texttt{ensure\_parent\_dirs\_for\_path} é um utilitário de OS.
\end{itemize}

\section{Execução de Processos}
O comando \texttt{execute\_process} implementa lógica complexa de criação de processos, redirecionamento de pipes (stdout/stderr) e tratamento de diferenças entre Windows (\texttt{CreateProcess}) e Unix (\texttt{system/fork}).

\textbf{Impacto:} Dificulta a portabilidade e testes unitários do avaliador sem efeitos colaterais no sistema.

\chapter{Lógica de Toolchain e Compilação}
O avaliador contém conhecimento indevido sobre como invocar compiladores específicos (MSVC vs GCC/Clang).

\section{Probes de Compilador}
As funções que implementam \texttt{check\_c\_source\_compiles} e \texttt{try\_compile} constroem linhas de comando manualmente.

\begin{itemize}
    \item \textbf{Hardcoded Flags:} O código verifica \texttt{probe\_compiler\_looks\_msvc} e decide entre usar \texttt{/Fe:} ou \texttt{-o}.
    \item \textbf{Gestão de Extensões:} O avaliador sabe que no Windows o objeto é \texttt{.obj} e no Linux é \texttt{.o}.
\end{itemize}

Esta lógica pertence a uma camada de abstração de \textbf{Toolchain Driver}, não ao interpretador da linguagem.

\chapter{Sub-parsers Embutidos}
O arquivo contém implementações completas de parsers para linguagens secundárias utilizadas dentro de strings do CMake.

\section{Parser Matemático}
O comando \texttt{math(EXPR ...)} desencadeia um parser de descida recursiva completo (\texttt{math\_parse\_add}, \texttt{math\_parse\_mul}, etc.) implementado dentro do arquivo de avaliação.

\section{Generator Expressions}
A função \texttt{eval\_generator\_expression} e seus auxiliares implementam um parser para a sintaxe \texttt{\$<...>}, incluindo aninhamento e balanceamento de parênteses.

\section{Manipulação de Strings e Regex}
Algoritmos complexos de substituição de expressões regulares (incluindo back-references \texttt{\textbackslash 1}) estão implementados inline.

\chapter{Recomendações de Refatoração}

Para corrigir os problemas acima, recomenda-se a seguinte decomposição do módulo:

\begin{enumerate}
    \item \textbf{Mover para \texttt{build\_model.c}:}
        \begin{itemize}
            \item Setters de propriedades e flags de targets.
            \item Gerenciamento de listas de \texttt{Custom\_Command}, \texttt{Tests} e \texttt{InstallRules}.
            \item Lógica de resolução de configurações (\texttt{DEBUG}, \texttt{RELEASE}).
        \end{itemize}
    
    \item \textbf{Criar \texttt{sys\_utils.c} (ou \texttt{os\_adapter.c}):}
        \begin{itemize}
            \item Abstrações para \texttt{copy\_file}, \texttt{recursive\_delete}, \texttt{make\_directory}.
            \item Abstração para execução de processos subprocessos.
        \end{itemize}
        
    \item \textbf{Criar \texttt{toolchain\_driver.c}:}
        \begin{itemize}
            \item Lógica para detectar compilador e gerar comandos de compilação (\texttt{try\_compile}).
        \end{itemize}
        
    \item \textbf{Extrair Parsers:}
        \begin{itemize}
            \item \texttt{math\_parser.c}: Para expressões matemáticas.
            \item \texttt{genex\_evaluator.c}: Para Generator Expressions.
        \end{itemize}
\end{enumerate}

\section{Conclusão}
A refatoração proposta reduzirá drasticamente o tamanho do \texttt{transpiler\_evaluator.inc.c}, transformando-o em um orquestrador focado puramente na semântica dos comandos CMake, delegando a execução e armazenamento para módulos especializados. Isso aumentará a manutenibilidade, testabilidade e clareza do código.

\end{document}